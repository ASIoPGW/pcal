% Chapter 1

\chapter{Introduction} % Main chapter title

\label{Chapter1} % For referencing the chapter elsewhere, use \ref{Chapter1} 

%----------------------------------------------------------------------------------------

% Define some commands to keep the formatting separated from the content 
\newcommand{\keyword}[1]{\textbf{#1}}
\newcommand{\tabhead}[1]{\textbf{#1}}
\newcommand{\code}[1]{\texttt{#1}}
\newcommand{\file}[1]{\texttt{\bfseries#1}}
\newcommand{\option}[1]{\texttt{\itshape#1}}

%----------------------------------------------------------------------------------------

\section{Calibration of KAGRA}
 
%----------------------------------------------------------------------------------------

\section{Requirement of KAGRA}
(1) GW 信号の符号の確認
フィードバック回路で、何重ものフィルターを介して反転を繰り返しているため、
検出した重力波信号の符号が正しいと確信するのは意外に難しい。
回路を一つでも見逃したり、極性を間違えると符号が反転するリスクがある。
一方で、万が一符号を間違えると物理結果には甚大な影響を及ぼす。
その点、Pcal は原理的に一方向しか押せないので、GW 信号の符号確認に適している。
実際、LIGO/Virgo ではこのような議論が度々起きており、
Pcal 導入の最初のモチベーションになっている

(2) 観測中でのキャリブレーション
Pcal が無い場合は、Laser wavelength などを使ったキャリブレーションが必要だが、
これは、観測状態では行えないので、キャリブレーションランから観測ランの
間にキャリブレーションパラメータが変わってしまうリスクをどのように
系統誤差に反映させるかが問題となる。一方で、Pcal があると、観測中及び、
観測感度を達成した状態でのキャリブレーションランが可能

(3) インジェクション
アクチュエータを使ったインジェクションでは、フィードバック回路のサーボの抑制により、
高周波でのインジェクションが難しくなる。その点、Pcal はアクチュエータよりも
効率よく高周波でのハードウェアインジェクションを可能にする

(4)キャリブレーション手法の独立性、干渉計の信頼性担保
制御ループにキャリブレーション信号を注入する従来の方法では、
いわば自分で自分を較正することになるので、あるレベル以上の系統
誤差は詰められない。
PCALの結果をだすことで、主干渉計の信頼性はぐっと増す。
従来の方法は、アクチュエータ効率の推定など、ループ内部にある誤
差・不定性の分離や評価が難しい。
PCALはそれを可能にする。PCAL自体にも誤差や不定性はあるが、制御
ループと独立な手段で基準信号を注入できることは意味がある。
手法が独立故に、系統誤差を減らす工夫が可能になるからだ。

(5)国際化
KAGRA, LIGO, Virgoで絶対値で正しい(お互いのズレを評価できる)
キャリブレーションが必要である。
ずれがあると、方向推定の誤差など、あらゆる物理をずらす。
このことは、例えば、宇宙線空気シャワー観測とX線観測の議論の違
いを見ると実感できる。標準エネルギーソースがない宇宙線空気シ
ャワー観測では常にそして長い間宇宙線のエネルギー推定の正しさを
巡って常に苦労している。一方、放射線源で較正が可能なX線観測で
は、検出したX線自体のエネルギーの中央値についての疑義はほとんど
でない。放射性同位体が較正線源として利用できる。重力実験では
とくに「絶対較正」が難しく、しかしそれが実験の要だったりする。
%----------------------------------------------------------------------------------------

\section{Schedule of KAGRA}


%----------------------------------------------------------------------------------------

