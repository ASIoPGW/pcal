% Chapter 1

\chapter{Introduction} % Main chapter title

\label{Chapter1} % For referencing the chapter elsewhere, use \ref{Chapter1} 

%----------------------------------------------------------------------------------------

% Define some commands to keep the formatting separated from the content 
\newcommand{\keyword}[1]{\textbf{#1}}
\newcommand{\tabhead}[1]{\textbf{#1}}
\newcommand{\code}[1]{\texttt{#1}}
\newcommand{\file}[1]{\texttt{\bfseries#1}}
\newcommand{\option}[1]{\texttt{\itshape#1}}

%----------------------------------------------------------------------------------------

\section{Calibration of a Gravitational-Wave detector}

KAGRA is a laser interferometer using four mirrors (test masses) suspended 
from multi-stage pendulums to form two perpendicular optical cavities (arms).
Gravitation Wave (GW) strain causes differential variations of the arm length
and generates power fluctuations in the detector readout port. 
The power fluctuations measured by photodetectors work as the GW readout 
signal and an error signal to control the differential arm length. 
For the stable operation of the instrument, a feedback control of the 
differential arm length is required. This control is achieved by taking 
a digitized readout signal, applying a set of digital filters, and sending 
the control signal to the test mass actuators. Therefore, estimation 
of the equivalent GW strain sensed by the interferometer requires 
detailed characterization and correction for the feedback control loop.

The uncertainties are directly translated to the systematic errors on 
the absolute GW signal amplitude. In case of LIGO GW150914 event, 
the calibration was established to an uncertainty (1$\sigma$) of less than 
10\% in amplitude and 10 degrees in phase~\cite{GW150914}.
In case of LIGO GW151226 event, the calibration uncertainty (1$\sigma$) 
in both detectors at the time of the signal is better than 8\% 
in amplitude and 5 degrees in phase~\cite{GW151226}.

The calibration uncertainties also affect the coordinate reconstruction 
particularly in the case that only up to three detectors in the world 
GW detector network can detect the GW signal. This can often happen 
because the sensitivity of interferometer has directional dependence. 
The effect of calibration uncertainties is visible at high signal-to-noise 
ratio events where the angular resolution is less affected by the detector 
noise. In such cases, the pointing accuracy can get worse by factor of 
2$\sim$4 with 10\% calibration uncertainties~\cite{Klimenko}.

%----------------------------------------------------------------------------------------

\newpage
\section{Roles of the photon calibrator}

Photon calibrators are the primary calibration tool in the Advanced LIGO 
and Advanced Virgo detectors~\cite{LIGO-CAL,Karki,Virgo-PCAL}.
Earlier versions have been tested on various 
interferometers~\cite{Accadia,Clubley:2001,Mossavi:2006},
and they have evolved significantly in LIGO over the past ten 
years~\cite{Goetz:2009}.
There are several unique roles required to the photon calibrator:

\begin{enumerate}
\item {\bf Check of the sign of \sl h(t)}\\
Since the direction of the movement of test mass is proportional to the 
laser power, photon calibrator allows a direct check of the sign of the 
reconstructed $h(t)$ channel compared to the definition taken in agreement 
with other experiments. In initial phase of Virgo, he primary purpose of 
photon calibrator was to check the sign of $h(t)$~\cite{VIR-018}.

\item {\bf Calibration during the observing periods}\\
Calibration methods without using photon calibrator such as using 
radio-frequency oscillator and laser wavelength can be done only under the 
limited condition where the interferometer is not operating in the optimum 
sensitivity. The propagation of calibration parameters from the high noise 
condition to the low noise condition can introduce additional unknown source 
of systematic errors. On the other hand, the photon calibrator is a completely 
independent instrument of the interferometer and therefore, actuate 
the test masses during the observation periods with optimum sensitivity. 

\item {\bf Hardware injection}\\
Hardware injection in the high frequency region is important to verify the 
response of the detector system. Since actuators are inside the feedback 
control loop, the amplitude is more suppressed for higher frequencies. 
Photon calibrator can inject the high frequency signal more efficiently. 

\item {\bf Independence of calibration method\\
      and reliability assurance of interferometer}\\
Injecting calibration signals into the control feedback loop has a limitation 
to reduce the systematic errors because it is calibrating the loop itself.
Without Pcal, it is difficult to disentangle each uncertainty inside the loop, 
such as optical gain and actuator efficiency. On the other hand, Pcal has a 
strong advantage to enable to inject calibration signals independent of the 
control loop and provide additional way to reduce the systematic uncertainties.

\item {\bf Globalization of the calibration}\\
It is necessary to calibrate and compare the absolute accuracies of KAGRA, 
LIGO and Virgo, or at least we need to have a way to evaluate the difference 
of absolute GW amplitude between different detectors. This kind of difference 
introduces bias on the physics parameters such as the source localization. 
Typical examples are cosmic-ray air shower observation and X-ray observations. 
In the long history of the air shower experiments, there have been always 
discussions about the absolute energy estimation. On the other hand, X-ray 
detector can be calibrated by the radio isotope sources and there is no 
question raised. In the GW experiment, absolute calibration is a difficult 
work but therefore it will be the key of the experimental technique. 

\end{enumerate}


%----------------------------------------------------------------------------------------

\section{Schedule of KAGRA}

In order to coincide the observation plan by LIGO and Virgo~\cite{LV-obs},
KAGRA is currently installing the instrument tomeet the following observation
schedule~\cite{KAGRA-obs}:

\begin{enumerate}
\item Phase-1: 2017.3 -- 2018.3\\
      Operation of Michelson interferometer in cryogenic condition 
      followed by introducing Fabry P\'{e}rot cabity.
\item Phase-2: 2018.4 -- 2019.3 (Opening)\\
      Full lock of RSE (Resonant Sideband Extraction)
\item Phase-3: 2019.4 -- 2020.3 (Early) \\
      One year commissioning after the first full operation, 
      then improve the sensitivities to achieve the design goal
\item 2020 -- 2021: Middle term
\item 2021 -- 2022: Late term

\end{enumerate}


%----------------------------------------------------------------------------------------

\section{Calibration requirements from data analysis}

%----------------------------------------------------------------------------------------

