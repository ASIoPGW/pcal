% Chapter 1

\chapter{Photon calibrator} % Main chapter title

\label{Chapter1} % For referencing the chapter elsewhere, use \ref{Chapter1} 

%----------------------------------------------------------------------------------------

% Define some commands to keep the formatting separated from the content 


%----------------------------------------------------------------------------------------

\section{Photon calibrator}
One of the goals of the gravitational wave experiment is the accurate measurement of the gravitational waveform that is measured through the absolute displacement of the end test masses. A recent study that LIGO conducted in US showed that a displacement uncertainty could be controlled  by the photon calibrator. The photon calibrator is one of the calibration tools to push the mirror surface using the photon pressure of the laser as shown in Fig XXX.
The absolute displacement is described as
\begin{equation}
dx=\frac{P \cos{\theta}}{c} s(f) \left( 1+\frac{I}{M}\vec{a}\cdot \vec{b}\right), \label{eq:dx}
\end{equation}
where $P$ is an absolute power of the laser, $c$ is the speed of light, $\theta$  is an incident angle of the laser, $I$ and $M$ are moment of inertia and mass of test mass, $\vec{a}$ and $\vec{b}$ is position vector of photon calibrator lasers and interferometer laser. Then, $s(f)$ is transfer function of the test masses. We simulated the transfer function of test mass as shown in Fig. XXX. We assumed the masses, shapes and Young's modules of the each pendulum mass and fiber as shown in Fig.XXX. According to transfer function, we can regard the motion of high frequency as free mass due to higher than the natural frequency. Therefore, we are able to assume as follows:
\begin{equation}
s(f)=\frac{1}{M \omega^2},
\end{equation}
where $\omega$ is the angular frequency of test mass.
 
%----------------------------------------------------------------------------------------

\section{Purpose of photon calibrator}
\subsection{Interferometer Calibration}
\subsection{Hardware injection test}
\subsection{Photon pressure actuator}

%----------------------------------------------------------------------------------------

\section{Calibration line of KAGRA}


%----------------------------------------------------------------------------------------

